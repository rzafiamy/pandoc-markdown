\documentclass{article}
\usepackage[utf8]{inputenc}
\providecommand{\tightlist}{\relax}

\usepackage{helvet}
\renewcommand{\familydefault}{\sfdefault}

\usepackage{titling} % Package for custom title formatting

\usepackage[a4paper, top=0.8in, right=0.5in, bottom=0.8in, left=0.5in]{geometry}

\usepackage{fancyhdr}

\pagestyle{fancy}
\fancyhf{} % Clear default header/footer

% Header settings
\fancyhead[C]{\textbf{}} % Header title in the center

% Footer settings
\renewcommand{\footrulewidth}{0.4pt} % Thickness of the footer line
\fancyfoot[L]{\textit{}} % Footer text on the left
\fancyfoot[C]{\thepage} % Page number at the center
\fancyfoot[R]{\today} % Date on the right


\begin{document}

\pretitle{\begin{center}\LARGE} % Adjust title formatting
\posttitle{\par\end{center}\vskip 0.5em} % Adjust title formatting

\title{}
\author{}
\date{\today}

\maketitle

\tableofcontents

\clearpage

\section{Heading 1}\label{heading-1}

Lorem ipsum dolor sit amet, consectetur adipiscing elit. Phasellus a
ullamcorper sapien.

\subsection{Heading 2}\label{heading-2}

Lorem ipsum dolor sit amet, consectetur adipiscing elit. Phasellus a
ullamcorper sapien.

\subsubsection{Heading 3}\label{heading-3}

Lorem ipsum dolor sit amet, consectetur adipiscing elit. Phasellus a
ullamcorper sapien.

\begin{Shaded}
\begin{Highlighting}[]
\KeywordTok{def}\NormalTok{ hello_world():}
    \BuiltInTok{print}\NormalTok{(}\StringTok{"Hello, World!"}\NormalTok{)}

\NormalTok{hello_world()}
\end{Highlighting}
\end{Shaded}

In this example, we have:

\begin{itemize}
\tightlist
\item
  Heading 1 denoted by a single hash symbol (\texttt{\#}).
\item
  Heading 2 denoted by two hash symbols (\texttt{\#\#}).
\item
  Heading 3 denoted by three hash symbols (\texttt{\#\#\#}).
\item
  A code block created using triple backticks
  (``\texttt{)\ with\ the\ language\ specifier}python\texttt{.\ Inside\ the\ code\ block,\ there\ is\ a\ simple\ Python\ function}hello\_world()`
  that prints ``Hello, World!''.
\item
  ``Lorem ipsum'' text used as placeholder content.
\end{itemize}

You can save this Markdown content in a file with a \texttt{.md}
extension, such as \texttt{example.md}, and then use Pandoc to convert
it to other formats like HTML, PDF, or LaTeX using the appropriate
command.

\end{document}

